\documentclass[a4paper]{article}

\usepackage[a4paper, top=2cm, bottom=2cm, left=2cm, right=2cm]{geometry}
\usepackage[T1, T2A]{fontenc}
\usepackage[fontsize=12]{fontsize}

\usepackage[english, russian]{babel}
\usepackage[utf8]{inputenc}

\usepackage{hyperref}
\usepackage{indentfirst}

\hypersetup{pdfpagemode=FullScreen, colorlinks=true, linkcolor=black, urlcolor=cyan}

\begin{document}
	\thispagestyle{empty}
	
	\begin{center}
	{\LARGE \textsc{НОВОСИБИРСКИЙ ГОСУДАРСТВЕННЫЙ УНИВЕРСИТЕТ}\par}
	{\textsc{ФАКУЛЬТЕТ ИНФОРМАЦИОННЫХ ТЕХНОЛОГИЙ}\par}
	
	\vspace{3cm}
	
	{\huge\bfseries Уголовное право и процесс\par}
	
	\vspace{1cm}
	
	{\Large\bfseries Уголовно-правовой анализ преступления\par}
	
	\vspace{10cm}
	
	\begin{flushright}
		Кондренко К.П, группа 21203
	\end{flushright}
	
	\vfill
	
	{\large \today\par}
\end{center}
	
	\newpage
	
	\tableofcontents
	
	\pagestyle{plain}
	
	\newpage
	
	\section{Возбуждение гражданского дела}
		14.02.2023 ООО "Югставнефть" обратилось в суд с иском, в котором просило взыскать с Мочалова А.В. сумму ущерба в размере 1 562 373 рубля 20 копеек, расходы по оплате госпошлины в размере 16 012 рублей.
		
		В обоснование заявленных требований истец указал, что <дата обезличена> между ООО "Югставнефть" и Мочаловым А.В. был заключен трудовой договор, этим же договором ответчик принял на себя полную материальную ответственность за недостачу вверенного ему имущества. <дата обезличена>Сочалов А.В. предоставил сведения об объеме на складе дизельного топлива 54,931 тонн и дизельного топлива евро 25,971 тонн. В то же время по бухгалтерскому учету числилось 59,939 тонн дизельного топлива и 25,120 тонн дизельного топлива евро. Таким образом, была выявлена недостача на складе в количестве 4,157 тонн дизельного топлива. Согласно акту инвентаризации товаров на складе от 16 февраля 2022 года в резервуаре <номер обезличен> имелось 49,420 тонн дизельного топлива и 25,120 тонн дизельного топлива евро. С учетом данных об остатки нефтепродуктов на складе ООО "Югставнефть" выставило контрагентам ООО "Агрофирме "КИЦ" и ООО "КАЦ" счета на предоплату. С 16 февраля 2022 по 18 февраля 2022 ответчик о других отгрузках не сообщал и они не предполагались. Согласно накладной б/н 19 февраля 2022 года Мочалов А.В. отгрузил водителю Пожидаеву 18,360 тонн вместо 25 тонн вместимости автомобиля, согласно накладной б/н 19 февраля 2022 года Мочалов А.В. отгрузил водителю Аверченко 21 тонну вместо 25 тонн вместимости автомобиля. 21 февраля 2022 года Мочалов А.В. подал заявление об увольнении по собственному желанию. 24 февраля 2022 года был издан приказ о прекращении действия трудового договора по инициативе работника. 25 февраля 2022 года была проведена инвентаризация, согласно которой была выявлена недостача. Мочалов А.В. был письменно извещен о выявленной недостаче и о необходимости явиться для участия в инвентаризации. В ответ он сообщил, что явится через полчаса, однако от явки уклонился. На письменное требование представить объяснения о недостаче, Мочалов А.В. письменное объяснение не представил, обвинив уполномоченное от имени ООО "Югставнефть" лицо в вымогательстве. По факту отказа в даче объяснений был составлен акт. В дальнейшем от каких-либо объяснений Мочалов А.В. уклонился.
		
	\section{Подготовка гражданского дела к судебному разбирательству}
		Истец ООО "Югставнефть" в судебное заседание не явился, ходатайствовал о рассмотрении дела в его отсутствие.
		
		Ответчик Мочалов А.В. в судебное заседание не явился. Дело рассмотрено в порядке заочного производства.
		
	\section{Судебное разбирательство}
		В судебном заседании установлено, что <дата обезличена> года между ООО "Югставнефть" и Мочаловым А.В. был заключен трудовой договор.
		
		Приказом <номер обезличен>к от <дата обезличена> Мочалов А.В. был принят на работу в ООО "Югставнефть" в должности заведующий складом нефтепродуктов.
		
		Также <дата обезличена> между сторонами был заключен договор о полной материальной ответственности.
		
		Приказом <номер обезличен>к от <дата обезличена> ответчик был уволен по инициативе работника.
		
		<дата обезличена> была проведена инвентаризация товарно-материальных ценностей.
		
		По итогам проведенной инвентаризации была выявлена недостача в размере 1 562 373, 24 рублей, что отражено в сличительной ведомости результатов инвентаризации товарно-материальных ценностей <номер обезличен> от <дата обезличена>.
		
		Согласно ст.11 Федерального закона от 06.12.2011 N 402-ФЗ "О бухгалтерском учете" активы и обязательства подлежат инвентаризации. При инвентаризации выявляется фактическое наличие соответствующих объектов, которое сопоставляется с данными регистров бухгалтерского учета.
		
		Приказом Министерства финансов Российской Федерации от 29 июля 1998 г. N 34н утверждено Положение по ведению бухгалтерского учета и бухгалтерской отчетности в Российской Федерации.
		
		Пунктами 26 и 28 названного положения установлено, что инвентаризация имущества и обязательств проводится для обеспечения достоверности данных бухгалтерского учета и бухгалтерской отчетности, в ходе ее проведения проверяются и документально подтверждаются наличие, состояние и оценка указанного имущества и обязательств. При этом выявленные при инвентаризации расхождения между фактическим наличием имущества и данными бухгалтерского учета отражаются на счетах бухгалтерского учета.
		
		Приказом Министерства финансов Российской Федерации от 13 июня 1995 г. N 49 утверждены Методические указания по инвентаризации имущества и финансовых обязательств (далее - Методические указания).
		
		По форме и содержанию ведомость соответствует Методическим указаниям, сведений об обратном суду не представлено.
		
		Проверка фактического наличия имущества проведена при участии материально ответственного лица Мочалова А.В., что подтверждается его подписью в акте инвентаризации товаров на складе нефтепродуктов от <дата обезличена>.
		
		Для проведения инвентаризации создана комиссия, в которую вошли директор (председатель), водитель и главный бухгалтер (члены комиссии). Инвентаризационная опись подписана всеми членами комиссии.
		
		Ответчик со своей стороны результаты инвентаризации не оспорил, не указал на какие-либо нарушения, допущенные при ее проведении.
		
		Согласно ч.1 ст.232 ТК РФ сторона трудового договора (работодатель или работник), причинившая ущерб другой стороне, возмещает этот ущерб в соответствии с Трудовым кодексом Российской Федерации и иными федеральными законами.
		
		Расторжение трудового договора после причинения ущерба не влечет за собой освобождение стороны этого договора от материальной ответственности, предусмотренной Трудовым кодексом Российской Федерации или иными федеральными законами (ч.3 ст.232 ТК РФ).
		
		Условия наступления материальной ответственности стороны трудового договора установлены ст.233 ТК РФ. В соответствии с этой нормой материальная ответственность стороны трудового договора наступает за ущерб, причиненный ею другой стороне этого договора в результате ее виновного противоправного поведения (действий или бездействия), если иное не предусмотрено данным кодексом или иными федеральными законами. Каждая из сторон трудового договора обязана доказать размер причиненного ей ущерба.
		
		Главой 39 ТК РФ "Материальная ответственность работника" определены условия и порядок возложения на работника, причинившего работодателю имущественный ущерб, материальной ответственности, в том числе и пределы такой ответственности.
		
		Работник обязан возместить работодателю причиненный ему прямой действительный ущерб. Неполученные доходы (упущенная выгода) взысканию с работника не подлежат (ч.1 ст.238 ТК РФ).
		
		Под прямым действительным ущербом понимается реальное уменьшение наличного имущества работодателя или ухудшение состояния указанного имущества (в том числе имущества третьих лиц, находящегося у работодателя, если работодатель несет ответственность за сохранность этого имущества), а также необходимость для работодателя произвести затраты либо излишние выплаты на приобретение, восстановление имущества либо на возмещение ущерба, причиненного работником третьим лицам (ч.2 ст.238 ТК РФ).
		
		Полная материальная ответственность работника состоит в его обязанности возмещать причиненный работодателю прямой действительный ущерб в полном размере (ч.1 ст.242 ТК РФ).
		
		Частью 2 ст.242 ТК РФ предусмотрено, что материальная ответственность в полном размере причиненного ущерба может возлагаться на работника лишь в случаях, предусмотренных этим кодексом или иными федеральными законами.
		
		Перечень случаев возложения на работника материальной ответственности в полном размере причиненного ущерба приведен в ст.243 ТК РФ.
		
		Так, в соответствии с п.2 ч.1 ст.243 ТК РФ материальная ответственность в полном размере причиненного ущерба возлагается на работника в случае недостачи ценностей, вверенных ему на основании специального письменного договора или полученных им по разовому документу.
		
		Материальная ответственность работника исключается в случаях возникновения ущерба вследствие непреодолимой силы, нормального хозяйственного риска, крайней необходимости или необходимой обороны либо неисполнения работодателем обязанности по обеспечению надлежащих условий для хранения имущества, вверенного работнику (ст.239 ТК РФ).
		
		В силу ч.1 ст.247 ТК РФ до принятия решения о возмещении ущерба конкретными работниками работодатель обязан провести проверку для установления размера причиненного ущерба и причин его возникновения. Для проведения такой проверки работодатель имеет право создать комиссию с участием соответствующих специалистов.
		
		Согласно ч.2 ст.247 ТК РФ истребование от работника письменного объяснения для установления причины возникновения ущерба является обязательным. В случае отказа или уклонения работника от предоставления указанного объяснения составляется соответствующий акт.
		
		В пункте 4 постановления Пленума Верховного Суда Российской Федерации от 16 ноября 2006 г. N 52 "О применении судами законодательства, регулирующего материальную ответственность работников за ущерб, причиненный работодателю" (далее - постановление Пленума Верховного Суда Российской Федерации от 16 ноября 2006 г. N 52) разъяснено, что к обстоятельствам, имеющим существенное значение для правильного разрешения дела о возмещении ущерба работником, обязанность доказать которые возлагается на работодателя, в частности, относятся: отсутствие обстоятельств, исключающих материальную ответственность работника; противоправность поведения (действий или бездействия) причинителя вреда; вина работника в причинении ущерба; причинная связь между поведением работника и наступившим ущербом; наличие прямого действительного ущерба; размер причиненного ущерба; соблюдение правил заключения договора о полной материальной ответственности.
		
		Истцом представлены доказательства соблюдения порядка принятия решения о возмещении ущерба ответчиком. У Мочалова А.В. были затребованы письменные объяснения по вопросу недостачи вверенных ему нефтепродуктов, от дачи которых он отказался, что подтверждается актом от <дата обезличена>, подписанным руководителем Семеновым А.Н., водителем Кизиловым В.Н., Тюриным А.В.
		
		Истцом также представлены доказательства размера причиненного ущерба, который подтверждается результатами инвентаризации. Также в ходе рассмотрения дела была получена справка об исследовании документов, составленная старшим специалистом-ревизором отделения <номер обезличен> <данные изъяты> <адрес обезличен> Кириченко Л.В., содержащая аналогичные сведения.
		
		Правила заключения договора о полной материальной ответственности соблюдены, суду представлен соответствующий договор, содержащий все существенные условия договора о полной материальной ответственности.
		
		Обстоятельства, исключающие материальную ответственность работника, судом не установлены.
		
		Вина работника заключается в необеспечении сохранности вверенных ему материальных ценностей.
		
		В силу ч.4 ст.392 ТК РФ работодатель имеет право обратиться в суд по спорам о возмещении работником ущерба, причиненного работодателю, в течение одного года со дня обнаружения причиненного ущерба.
		
		Истец обратился в суд с рассматриваемым иском 14 февраля 2023 года, то есть до истечения указанного срока.
		
		При таких обстоятельствах требование ООО "Югставнефть" о взыскании причиненного Мочаловым А.В. ущерба подлежит удовлетворению.
		
		В соответствии с требованиями ст. 98 ГПК РФ, стороне, в пользу которой состоялось решение суда, суд присуждает возместить с другой стороны все понесенные по делу судебные расходы.
		
		Согласно ст. 88 ГПК РФ, судебные расходы состоят из государственной пошлины и издержек, связанных с рассмотрением дела.
		
		При подаче искового заявления истцом была уплачена государственная пошлина в размере 16 012 рублей, что подтверждается платежным поручением <номер обезличен> от 09 февраля 2023 года.
		
		Поскольку требования истца удовлетворены в полном объеме, с ответчика в пользу истца подлежат взысканию расходы по оплате государственной пошлины в размере 16 012 рублей.
	
	\section{Судебное решение}
		На основании изложенного, руководствуясь ст.194-198 ГПК РФ, суд решил исковые требования Общества с ограниченной ответственностью "Югставнефть" – удовлетворить.
		
		Взыскать с Мочалова А. В. (<дата обезличена> года рождения, паспорт <номер обезличен>) в пользу общества с ограниченной ответственностью "Югставнефть" ( ИНН <номер обезличен>, ОГРН <номер обезличен>) ущерб, причиненный работодателю, в размере 1 562 373 рубля 20 копеек, судебные расходы по уплате государственной пошлины в размере 16 012 рублей.
	
	\section{Стадия апелляционного обжалования постановлений суда}
		19.07.2023 было зарегистрировано ходатайство/заявление (предположительно ответчика) о вынесении дополнительного решения. 28.07.2023 поступившее ходатайство/заявление было изучено и назначено судебное заседание для рассмотрения ходатайства/заявления/вопроса. 11.08.2023 Ходатайство/заявление удовлетворено и был начат пересмотр иска.
		
		Было взято во внимание, что инвентаризация поведенная 16.02.2022г. в бухгалтерском учете, на которой основаны претензии истца, не отражена - в ведомости счета 41 отсутствуют корреспондирующий счет 94 "Недостачи и потери от порчи ценностей" (Методические указания по бухгалтерскому учету материально-производственных запасов, утвержденных приказом Минфина России от 28.12.2001 N 119н).
		
		Следовательно, подтвердить проведенную 25 февраля 2022 года инвентаризацией недостачу материальных ценностей на складе нефтепродуктов ООО "Югставнефть" не представляется возможным.
			
		Проведенная 25 февраля 2022 года инвентаризация не соответствует требованиям Методических указаний по инвентаризации имущества и финансовых обязательств, утвержденных Приказом Минфина РФ от 13.06.1995 №49, Положения по ведению бухгалтерского учета и бухгалтерской отчетности в Российской Федерации, утвержденным Приказ Минфина России от 29.07.1998 №34н, Федерального закона от 06.12.2011 №402-ФЗ "О бухгалтерском учете" в части:
		
		\begin{enumerate}
			\item Отсутствие проведенной обязательной инвентаризации товарно-материальных ценностей при смене материально-ответственных сотрудников нарушает: \begin{itemize}
				\item  п.1.15 Приказа Минфина РФ от 13.06.1995 N 49 (ред. от 08.11.2010) "Об утверждении Методических указаний по инвентаризации имущества и финансовых обязательств"

				\item п. 27 Положения по ведению бухгалтерского учета и бухгалтерской отчетности в Российской Федерации, утвержденным Приказ Минфина России от 29.07.1998 №34н
				
				\item  п. 3 ст. И Федерального закона от 06.12.2011 № 402-ФЗ "О бухгалтерском учете"
			\end{itemize}
			
			\item Отсутствие письменного уведомления уволенного работника Молчанова А.В. о проведении инвентаризации с предложением принять в ней участие и отсутствие акта об отсутствии уволенного работника Молчанов А.В. при проведении инвентаризации с указанием причин его отсутствия свидетельствуют о нарушении: \begin{itemize}
				\item п.2.8, п.2.10, п.3.17 Приказа Минфина РФ от 13.06.1995 N 49 (ред. от 08.11.2010)    "Об утверждении Методических указаний по инвентаризации имущества и финансовых обязательств"
			\end{itemize}
			
			\item Отсутствие подписи материально ответственного лица (уволенного работника Молчанова А.В.) свидетельствуют о нарушении: \begin{itemize}
				\item п.2.8, п.2.10, п.3.17 Приказа Минфина РФ от 13.06.1995 N 49 (ред. от 08.11.2010)    "Об утверждении Методических указаний по инвентаризации имущества и финансовых обязательств"
			\end{itemize}
			
			\item Отсутствие в строке 3 таблицы инвентаризационной ведомости (по "Топливу низкозастыающему") в графе "Фактическое наличие" заполнения как количества так и суммы свидетельствуют о нарушении: \begin{itemize}
				\item п.2.9 Приказа Минфина РФ от 13.06.1995 N 49 (ред. от 08.11.2010) "Об утверждении Методических указаний по инвентаризации имущества и финансовых обязательств" - не допускается оставлять незаполненные строки, на последних страницах незаполненные строки прочеркиваются.
			\end{itemize}
		\end{enumerate}
		
		Также согласно выводам эксперта первичная документация по учету материальных ценностей (нефтепродуктов) имеющаяся в материалах гражданского дела, ведомость учета нефтепродуктов являются первичными учетными документами оформленными в соответствии с п.1 ст.9 Федерального закона от 06.12.2011г. №402-ФЗ "О бухгалтерском учете" - соблюдаются необходимые условия для принятия их к учету.
		
		Заключение проведенной по делу судебной экспертизы не оспорено, ходатайств о назначении повторной судебной экспертизы сторонами не заявлено.
		
		Допрошенная в судебном заседании с участием эксперта в качестве свидетеля Пономаренко Ю.А., работавшая бухгалтером в ООО "Югставнефть" на момент проведения инвентаризации, дала пояснения, касающиеся фактических обстоятельства ее проведения.
			
		Опрошенная в судебном заседании эксперт Леонова А.Р., предупрежденная об уголовной ответственности за дачу заведомо ложного заключения, выводы судебной экспертизы поддержала в полном объеме, при этом указала, что сообщенные Пономаренко Ю.А. сведения при ответе на вопросы ответчика никак не влияют на сделанные ею выводы.
			
		При изложенных обстоятельствах и установленных судом нарушений со стороны работодателя в ходе проведения инвентаризации, с учетом результатов проведенной по делу судебной экспертизы, принимая во внимание, что что факт недостачи может считаться установленным только при условии выполнения в ходе инвентаризации всех необходимых проверочных мероприятий, результаты которых должны быть оформлены документально в установленном законом порядке, суд приходит к выводу об отсутствии оснований для удовлетворения исковых требований.
			
		Также судом установлено, что <дата обезличена> следственным отделом <данные изъяты> возбуждено уголовное дело по факту хищения дизельного топлива со склада нефтепродуктов ООО "Югставнефть" неустановленным лицом на общую сумму 1 562 373, 24 рублей.
		
		Имеющаяся в уголовном деле справка <номер обезличен> от <дата обезличена> об исследовании документов, составленная старшим специалистом-ревизором отделения <данные изъяты> Кириченко Л.В., содержит описание представленных на исследование документов, а также вывод об отсутствии подписи материально-ответственного лица в инвентаризационных документах.
		
		В ходе судебного разбирательства представитель истца пояснил, что до настоящего времени лицо, причастное к преступлению, не установлено, обвинение ответчику не предъявлялось.
		
		Оценив представленные в материалы дела доказательства по правилам статьи 67 ГПК РФ, в том числе результаты проведенной по делу судебной бухгалтерской экспертизы, суд приходит к выводу об отсутствии доказательств, подтверждающих причинение ООО "Югставнефть" действиями Мочалова А.В. ущерба на заявленную в исковом заявлении сумму 1 562 373, 24 рублей.
		
		При рассмотрении дела о возмещении причиненного работодателю прямого действительного ущерба в полном размере работодатель обязан представить доказательства, свидетельствующие о том, что в соответствии с Трудовым кодексом Российской Федерации либо иными федеральными законами работник может быть привлечен к ответственности в полном размере причиненного ущерба (пункт 8 постановления Пленума Верховного Суда Российской Федерации от 16 ноября 2006 г. N 52).
		
		По смыслу вышеприведенных норм, работодатель, предъявляя требования о взыскании с работников причиненного ущерба на основании договора о полной материальной ответственности, обязан доказать размер ущерба и причину его возникновения, а также период возникновения ущерба для определения круга ответственных лиц, работавших в указанный период.
		
		Закрепляя право работодателя привлекать работника к материальной ответственности, Трудовой кодекс Российской Федерации предполагает, в свою очередь, предоставление работнику адекватных правовых гарантий защиты от негативных последствий, которые могут наступить для него в случае злоупотребления со стороны работодателя при его привлечении к материальной ответственности.
		
		При изложенных обстоятельствах и установленных судом нарушений со стороны работодателя в ходе проведения инвентаризации, с учетом результатов проведенной по делу судебной экспертизы, принимая во внимание, что что факт недостачи может считаться установленным только при условии выполнения в ходе инвентаризации всех необходимых проверочных мероприятий, результаты которых должны быть оформлены документально в установленном законом порядке, суд приходит к выводу об отсутствии оснований для удовлетворения исковых требований.
			
		\textbf{Решение:} На основании изложенного, руководствуясь ст.194-198 ГПК РФ, суд решил в удовлетворении исковых требований общества с ограниченной ответственностью "Югставнефть" к Мочалову Андрею Вадимовичу о взыскании ущерба, причиненного работодателю, судебных расходов – отказать.
			
	\section{Общий анализ}
		ООО "Югставнефть" подало иск на Мочалова А.В по факту материальной ответственности и недостачи на рабочем месте, ссылаясь на ТК РФ и подписанный трудовой договор с Мочаловым. Решением стало взыскание с Мочалова А.В соответствующей суммы, однако данное решение было обжаловано, и началось новое гражданское дело, в ходе которого было указано на несостоятельность исковых обвинений, так как они не соответствовали федеральным законам о бухгалтерском учёте, как следствие суд отказал ООО "Югставнефть" в удовлетворении исковых требований против Мочалова А.В.
		
	\section{Ссылки}
		\begin{itemize}
			\item \href{https://lenynsky--stv.sudrf.ru/modules.php?name=sud_delo&srv_num=1&name_op=case&n_c=1&case_id=393221682&case_uid=dbca7bee-9952-41e9-a827-c492f71cf459&delo_id=1540005&new=0}{Первое дело};
			
			\item \href{https://lenynsky--stv.sudrf.ru/modules.php?name=sud_delo&srv_num=1&name_op=case&case_id=393275451&case_uid=d4cac6ef-7851-4aa5-9c3a-69b1daea27ca&delo_id=1540005}{Второе дело}.
		\end{itemize}
\end{document}
