\documentclass[a4paper]{article}

\usepackage[a4paper, top=2cm, bottom=2cm, left=2cm, right=2cm]{geometry}
\usepackage[T1, T2A]{fontenc}
\usepackage[fontsize=12]{fontsize}

\usepackage[english, russian]{babel}
\usepackage[utf8]{inputenc}

\usepackage{hyperref}
\usepackage{indentfirst}

\hypersetup{pdfpagemode=FullScreen, colorlinks=true, linkcolor=black, urlcolor=cyan}

\begin{document}
	\thispagestyle{empty}
	
	\begin{center}
	{\LARGE \textsc{НОВОСИБИРСКИЙ ГОСУДАРСТВЕННЫЙ УНИВЕРСИТЕТ}\par}
	{\textsc{ФАКУЛЬТЕТ ИНФОРМАЦИОННЫХ ТЕХНОЛОГИЙ}\par}
	
	\vspace{3cm}
	
	{\huge\bfseries Уголовное право и процесс\par}
	
	\vspace{1cm}
	
	{\Large\bfseries Уголовно-правовой анализ преступления\par}
	
	\vspace{10cm}
	
	\begin{flushright}
		Кондренко К.П, группа 21203
	\end{flushright}
	
	\vfill
	
	{\large \today\par}
\end{center}
	
	\newpage
	
	\tableofcontents
	
	\pagestyle{plain}
	
	\newpage
	
	\section{Фабула преступлений}
		В период времени с 20.05.2013 по 02.09.2016 Соколовский, поддерживающий экстремистские взгляды на почве личных националистических взглядов, а также взглядов по отношению к религии, отдельным социальным группам совершил ряд умышленных преступлений, используя зарегистрированную персональную страницу в социальной интернет-сети <<Вконтаке>> <<Руслан Соколовский>>, в социальной интернет-сети <<Youtube>> персональную страницу пользователя <<Соколовский>>, а именно совершил действия, направленные на возбуждение ненависти либо вражды, на унижение достоинства человеком группы лиц по признаку национальности, отношения к религии, равно принадлежности к какой-либо социальной группе, публично с использованием информационно-телекоммуникационной сети <<Интернет>>, кроме того публичные действия, выражающие явное неуважение к обществу, совершённые в целях оскорбления религиозных чувств верующих, кроме того незаконно приобрёл специальное техническое средство, предназначенное для негласного получения информации, при невыясненных обстоятельствах.

		Соколовский, находясь на территории города Шадринска, Курганской области, используя техническое устройство, подключенное к информационно-телекоммуникационной сети <<Интернет>>, обладая познаниями в компьютерной технике и программном обеспечении, навыками использовании услуг сети <<Интернет>>, социальной сетью <<Youtube>>, умышленно с целью пропаганды информации, направленной на возбуждение ненависти либо вражды, на унижение достоинства человека, группы лиц по признаку национальности, отношения к религии, равно принадлежности к какой-либо социальной группе, совершил публичные действия: разместил в свободном доступе для неограниченного круга лица на интернет-странице пользователя <<Соколовский>> в социальной интернет-сети <<Youtube>> видеоролики с названиям
		\begin{enumerate}
			\item <<В космос летал --- чеченцев не видел>>,
			
			\item <<Письма ненависти. Феминистки>>,
			
			\item <<Вступил в секту>>,
			
			\item <<Суицид мусульман на ЕГЭ>>,
			
			\item <<Патриарх Кирилл ты>>,
			
			\item <<Ловим покемонов в церкви. Pokemon Go>>, записанное в помещении <<Храм всех святых земли российской просиявшей>> в городе Екатеринбурге,
			
			\item <<Идеальный православный брак>>,
		\end{enumerate}
		
		которые ранее самостоятельно изготовил, используя специальное техническое устройство с функций аудио и видео- записи. В ходе осмотра предметов, данная информация была обнаружена и скопирована на компакт-диск сотрудниками правоохранительных органов. Согласно заключению экспертов, в указанном виде видеоролики содержат признак, характеризующий возбуждение национальной или религиозной вражды с точки зрения общественной опасности, формируемый негативный образ группы лиц по религиозному признаку, по национальному, по социальному, также содержатся признаки возбуждения ненависти к верующим: в исследуемых видеороликах присутствуют лингвистические признаки унижения группы лиц, выделенной по религиозному признаку, а также оскорбления чувств верующих, при помощи неуместной метафоры <<зомби>>, которая в данном случае обозначает Иисуса, присутствует информация, содержащая в себе признаки оскорбления чувств приверженцев Христианства и Ислам, формируемого через представление и наделение Иисуса Христа качества ожившего мертвеца <<зомби>>, а также назывании Иисуса, <<редким покемоном>> в видеоролике <<Ловим покемонов в церкви. Pokemon Go>>.
		 
		Кроме того, Соколовский незаконно приобрёл специальное техническое средство, предназначенное для негласного получения информации, при следующих обстоятельствах: в период с 3-го марта 2016 года по 2-ое сентября 2016 года Соколовский, находясь в неустановленном месте при неустановленных обстоятельствах, не имея лицензии, выдаваемой органами федеральной службы безопасности РФ, то есть незаконно приобрёл устройство, смонтированное в корпусе шариковой авторучки, являющееся специальным техническим средством, предназначенным для негласного получения информации, что подтверждается заключением эксперта, согласно которому представленное на экспертизу устройство является фото и видео- регистратором, смонтированным в корпусе шариковой ручки, находящемся в работоспособном состоянии и соответствующем категории специальных технических средств, предназначенных для негласного получения визуальной и аудио-акустической информации, которое хранил по месту фактического проживания до момента его обнаружения и изъятия в ходе обыска, который был проведён 2-го сентября 2016 года. Таким образом - 138.1
	
		В судебном заседании Соколовский вину не признал, пояснив, что не имел намерений либо умысла на оскорбление религиозных чувств верующих, не совершал действий, направленных на возбуждение ненависти, либо вражды, а также не приобретал специальное техническое средство авторучку, данный предмет оставил в его квартире знакомый Сергей Лазарев, который находился в гостях. Аналогичные ручки можно приобрести на <<AliExpress>>, в ручке установлен световой индикатор, поэтому скрыть факт записи невозможно. Пытался провести такую ручку, но в итоге так и не приобрёл.
		
	\section{Квалификация преступлений}
	
		\begin{itemize}
			\item Преступления, совершённые по статье 138.1
			
			\begin{enumerate}
				\item \textbf{Родовой объект} --- против личности;
				
				\item \textbf{Видовой объект} --- против конституционных прав и свобод человека и гражданина;
				
				\item \textbf{Непосредственный} --- незаконный оборот специальных технических средств, предназначенных для негласного получения информации.
			\end{enumerate}
			
			\item Преступления, совершённые по статье 148
			
			\begin{enumerate}
				\item \textbf{Родовой объект} --- против личности;
				
				\item \textbf{Видовой объект} --- против конституционных прав и свобод человека и гражданина;
				
				\item \textbf{Непосредственный} --- нарушение права на свободу совести и вероисповеданий.
			\end{enumerate}
			
			\item Преступления, совершённые по статье 282
			
			\begin{enumerate}
				\item \textbf{Родовой объект} --- против государственной власти;
				
				\item \textbf{Видовой объект} --- против основ конституционного строя и безопасности государства;
				
				\item \textbf{Непосредственный} --- возбуждение ненависти либо вражды, а равно унижение человеческого достоинства.
			\end{enumerate}	
		\end{itemize}

	\section{Состав преступлений}
		Составы всех преступлений содержат один и тот же \textbf{субъект} --- Соколовский Р.Г. На момент преступления ему исполнилось 22 года. Официально безработный, ранее не судим, на учете у психиатра и нарколога не состоит, не привлекался к административной ответственности, вменяем.
	
		\begin{itemize}
			\item Преступления, совершённые по статье 138.1
			
			\begin{enumerate}
				\item \textbf{Объект} --- конституционные права и свободы человека и гражданина.
				
				\item \textbf{Объективная сторона} --- незаконное приобретение специальных технических средств, предназначенных для негласного получения информации.
				
				\item \textbf{Субъективная сторона} --- Соколовский осозновал общественную опасность приобретения специального технического средства, предвидел возможность наступления общественно-опасных последствий и желал их наступления, потому что нельзя купить что-то не желая этого.
			\end{enumerate}
			
			\item Преступления, совершённые по статье 148
			
			\begin{enumerate}
				\item \textbf{Объект} --- право на свободу совести и вероисповеданий.
				
				\item \textbf{Объективная сторона} ---  публичные действия, выражающие явное неуважение к обществу и совершенные в целях оскорбления религиозных чувств верующих.
				
				\item \textbf{Субъективная сторона} --- Соколовский выпустил несколько роликов на эту тематику, поэтому каждый раз возникал прямой умысел, состоявший из придумывания идеи ролика, написания сценария, съёмки, монтажа, выкладывания в <<Интернет>>. Соколовский признался, что делал это для привлечения внимания аудитории и последующего зарабатывания денег.
			\end{enumerate}
			
			\item Преступления, совершённые по статье 282
			
			\begin{enumerate}
				\item \textbf{Объект} --- нарушение конституционного принципа недопустимости экстремизма, как деяния, направленного на возбуждение ненависти или вражды.
				
				\item \textbf{Объективная сторона} --- совершение действий, направленных на возбуждение ненависти либо вражды, а также на унижение достоинства человека либо группы лиц по признакам пола, расы, национальности, языка, происхождения, отношения к религии, а равно принадлежности к какой-либо социальной группе, совершенные публично, в том числе с использованием средств массовой информации либо информационно-телекоммуникационных сетей, включая сеть "Интернет".
				
				\item \textbf{Субъективная сторона} --- Соколовский выпустил несколько роликов на эту тематику, поэтому каждый раз возникал прямой умысел, состоявший из придумывания идеи ролика, написания сценария, съёмки, монтажа, выкладывания в <<Интернет>>. Соколовский признался, что делал это для привлечения внимания аудитории и последующего зарабатывания денег.
			\end{enumerate}

		\end{itemize}
	
	\section{Стадии совершения преступлений}
	
		Данные преступления имеют лишь формальный состав, поэтому все преступления являются оконченными.
	
	\section{Возможное соучастие}
		
		Сергей Лазарев --- знакомый Соколовского, который возможно помог в незаконном приобретении специальных технических средств, предназначенных для негласного получения информации.
		
	\section{Обстоятельства, смягчающие наказание}
		
		\begin{enumerate}
			\item Руслан Соколовский ранее не судим;
			
			\item Руслан Соколовский имеет положительную характеристику, выданную суду его матерью, которой он оказывает материальную поддержку;
			
			\item Поведение в суде Руслана Соколовского, который просил прощения у всех, кого он оскорбил и сказал, что переосмыслил свои действия.
		\end{enumerate}
		
	\section{Уголовный процесс}
		Информационное агенство Ура.ру обратилось к правоохранительным органам с просьбой провести проверку на предмет наличия в высказываниях блогера Соколовского признаков преступления, предусмотренного статьей 148 УК РФ. 
		
		В отношении Соколовского было возбуждено уголовное дело по статье 148 УК РФ. 2 сентября Соколовский был арестован и помещён в СИЗО сроком на 2 месяца. 8 сентября Соколовский был переведён под домашний арест, однако вскоре был возвращён в СИЗО из-за нарушений условий ареста, так как к нему пришла его девушка, чтобы поздравить с днём рождения. 13 февраля 2017 года Соколовский был возвращён под домашний арест, а его дело было передано в Верх-Исетский районный суд Екатеринбурга.
		
		В ходе расследования уголовного дела по данному делу были проведены:
		\begin{itemize}
			\item Просмотр страниц Соколовского в социальных интернет-сетях <<Вконтакте>> и <<Youtube>> и копирование видео-материалов оттуда;

			\item Обыск по адресу фактического проживания Соколовского;

			\item Допрос Соколовского;

			\item Комплексная психолого-лингвистически-религиоведческо-социлогическая экспертиза, установившая наличие признаков вражды и ненависти в видеороликах Соколовского.
		\end{itemize}
		
		8 апреля в суде состоялись прения сторон. Обвинение попросило признать Соколовского виновным по всем пунктам обвинения и назначить наказание в виде 3.5 лет лишения свободы в колонии общего режима. Адвокат Бушмаков заявил, что уверен в полной невиновности подзащитного. В последнем слове Соколовский отказался признать свою вину, повторив, что он является атеистом, космополитом и либертарианцем, а опубликованные им ролики не преследовали цель оскорбления кого-либо по признаку религии или национальности.
			
	\section{Приговор}
	
		11 мая 2017 года суд Екатеринбурга признал Соколовского виновным в возбуждении вражды, оскорблении чувств верующих, а также в незаконном обороте специальных технических средств и приговорил его к 3.5 годам лишения свободы условно с тремя годами испытательного срока; предписал осуждённому удалить все видеоролики, которые оскорбляют чувства верующих, и запретил участвовать в массовых мероприятиях.
	
	\section{Ссылки}
		
		\begin{itemize}
			\item \href{https://verhisetsky--svd.sudrf.ru/modules.php?name=sud_delo&srv_num=2&name_op=case&case_id=288607166&case_uid=65bfc677-ea10-4cb5-8357-d0bce09c973c&delo_id=1540006&new=}{Приговор};
			
			\item \href{https://www.youtube.com/watch?v=s2pY4zCnUFQ&t=3093s}{Cуд};
			
			\item \href{https://www.youtube.com/watch?v=xlLUFwO1hbY&t=67s}{Совершение одного из преступлений};
			
			\item \href{https://www.youtube.com/watch?v=hEg0MCSr098}{Показания свидетельницы};
			
			\item \href{https://www.youtube.com/watch?v=0zZ07pV9s1g}{Последнее слово осуждённого}.
		\end{itemize}
		
\end{document}
