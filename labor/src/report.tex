\documentclass[a4paper]{article}

\usepackage[a4paper, top=2cm, bottom=2cm, left=2cm, right=2cm]{geometry}
\usepackage[T1, T2A]{fontenc}
\usepackage[fontsize=12]{fontsize}

\usepackage[english, russian]{babel}
\usepackage[utf8]{inputenc}

\usepackage{hyperref}
\usepackage{indentfirst}

\hypersetup{pdfpagemode=FullScreen, colorlinks=true, linkcolor=black, urlcolor=cyan}

\begin{document}
	\thispagestyle{empty}
	
	\begin{center}
	{\LARGE \textsc{НОВОСИБИРСКИЙ ГОСУДАРСТВЕННЫЙ УНИВЕРСИТЕТ}\par}
	{\textsc{ФАКУЛЬТЕТ ИНФОРМАЦИОННЫХ ТЕХНОЛОГИЙ}\par}
	
	\vspace{3cm}
	
	{\huge\bfseries Уголовное право и процесс\par}
	
	\vspace{1cm}
	
	{\Large\bfseries Уголовно-правовой анализ преступления\par}
	
	\vspace{10cm}
	
	\begin{flushright}
		Кондренко К.П, группа 21203
	\end{flushright}
	
	\vfill
	
	{\large \today\par}
\end{center}
	
	\newpage
	
	\tableofcontents
	
	\pagestyle{plain}
	
	\newpage
	
	\section{Основные положения}
		В данном разделе определяются субъекты, участвующие в этом договоре: работник и работодатель. \textbf{Работодатель} --- лицо, которое выдало трудовой договор, \textbf{работник} --- второе лицо.
	
	\section{Предмет договора}
		Этот раздел определяет цель заключения договора. Так как договор является трудовым договором с программистом, то предметом договора является труд, то есть выполнение трудовых обязанностей работника в должности программиста. Также здесь указаны условия работы и обязанность работника не разглашать коммерческую тайну.
	
	\section{Срок действия договора}
		Данный раздел определяет:
		\begin{enumerate}
			\item дату вступления договора в силу  (со дня его заключения, либо со дня фактического допущения Работника к работе);
			
			\item дату окончания действия силы договора;
			
			\item испытательный срок для сотрудника.
		\end{enumerate}
	\section{Условия оплаты труда работника}
		Здесь определены:
		\begin{enumerate}
			\item размер должностного оклада;
			
			\item документ, регулирующий стимулирующие и компенсационные выплаты;
			
			\item доплата за работу сверхурочно (первые два часа работы в полуторном размере, за последующие часы - в двойном размере);
			
			\item доплата за работу в выходной и нерабочий праздничный день (в размере одинарной части должностного оклада за день или час работы сверх должностного оклада, если работа в выходной или нерабочий праздничный день производилась в пределах месячной нормы рабочего времени, и в размере двойной части должностного оклада за день или час работы сверх должностного оклада, если работа производилась сверх месячной нормы рабочего времени);
			
			\item оплата за время простоя (2/3 от должностного оклада пропорционально времени простоя);
			
			\item обстоятельства удержания заработной платы (предусмотренны действующим законодательством Российской Федерации).
		\end{enumerate}
	\section{Отпуск. Режим рабочего времени и времени отдыха}
		Раздел определяет следующее:
		\begin{enumerate}
			\item режим рабочего времени и количество выходных в неделю;
			
			\item времена начала и окончания работы;
			
			\item время на перерыв для отдыха и питания в течение рабочего дня (в рабочее время не включается);
			
			\item длительность оплачиваемого отпуска (не менее 28 дней);
			
			\item порядок использования отпуска (за первый год работы --- через 6 месяцев, далее --- в любое время в соответствии с графиком отпусков);
			
			\item получение отпуска по семейным обстоятельствам и другим уважительным причинам;
			
			\item ежегодный дополнительный оплачиваемый отпуск (по необходимости).
		\end{enumerate}
	\section{Права, обязанности, трудовые функции работника}
		\textbf{Права}
		\begin{enumerate}
			\item предоставление работы, обусловленной трудовым договором;
			
			\item рабочее место, соответствующее государственным нормативным требованиям охраны труда и условиям, предусмотренным коллективным договором;
			
			\item своевременная и в полном объеме выплата заработной платы в соответствии со своей квалификацией, сложностью труда, количеством и качеством выполненной работы;
			
			\item  полная достоверная информацию об условиях труда и требованиях охраны труда на рабочем месте;
			
			\item профессиональная подготовка, переподготовка и повышениа своей квалификации в порядке, установленном Трудовым кодексом Российской Федерации, иными федеральными законами;
			
			\item объединение, включая право на создание профессиональных союзов и вступление в них для защиты своих трудовых прав, свобод и законных интересов;
			
			\item ведение коллективных переговоров и заключение коллективных договоров и соглашений через своих представителей, а также на информацию о выполнении коллективного договора, соглашений;
			
			\item защита своих трудовых прав, свобод и законных интересов всеми не запрещенными законом способами;
			
			\item разрешение индивидуальных и коллективных трудовых споров, включая право на забастовку, в порядке, установленном Трудовым кодексом Российской Федерации, иными федеральными законами;
			
			\item возмещение вреда, причиненного ему в связи с исполнением трудовых обязанностей, и компенсацию морального вреда в порядке, установленном Трудовым кодексом Российской Федерации, иными федеральными законами;
			
			\item обязательное социальное страхование в случаях, предусмотренных федеральными законами.
		\end{enumerate}
	
		\textbf{Обязанности}
		\begin{enumerate}
			\item разработка и отладка программного кода
			\begin{itemize}
				\item формализация и алгоритмизация поставленных задач;
				
				\item написание программного кода с использованием языков программирования, определения и манипулирования данными;
				
				\item оформление программного кода в соответствии с установленными требованиями;
				
				\item работа с системой контроля версий;
				
				\item проверка и отладка программного кода;
			\end{itemize}
			
			\item проверка работоспособности и рефакторинг кода программного обеспечения
			\begin{itemize}
				\item разработка процедур проверки работоспособности и измерения характеристик программного обеспечения;
				
				\item разработка тестовых наборов данных;
				
				\item проверка работоспособности программного обеспечения;
				
				\item рефакторинг и оптимизация программного кода;
				
				\item исправление дефектов, зафиксированных в базе данных дефектов;
			\end{itemize}
			
			\item интеграция программных модулей и компонент и верификация выпусков программного продукта
			\begin{itemize}
				\item разработка процедур интеграции программных модулей;
				
				\item осуществление интеграции программных модулей и компонент и верификации выпусков программного продукта;
			\end{itemize}
			
			\item разработка требований и проектирование программного обеспечения
			\begin{itemize}
				\item анализ требований к программному обеспечению;
				
				\item разработка технических спецификаций на программные компоненты и их взаимодействие;
				
				\item проектирование программного обеспечения;
			\end{itemize}
			
			\item соблюдение правил внутреннего трудового распорядка, трудовой дисциплины и требований по охране труда и обеспечению безопасности труда;
			
			\item бережно относиться к имуществу Работодателя и других работников;
			
			\item принимать необходимые меры и незамедлительно сообщать Работодателю либо непосредственному руководителю о возникновении ситуации, представляющей угрозу жизни и здоровью людей, сохранности имущества Работодателя;
			
			\item не разглашать сведений, составляющих коммерческую тайну Работодателя (определены в Положении о коммерческой тайне);
			
			\item по распоряжению Работодателя отправляться в служебные командировки на территории России и за рубеж.
		\end{enumerate}
	
	\section{Социальное страхование работника}
		Работник подлежит обязательному социальному страхованию от несчастных случаев на производстве и профессиональных заболеваний в порядке и на условиях, установленных действующим законодательством РФ.
	
	\section{Гарантии и компенсации}
		Никаких дополнительных гарантий и компенсаций не предусмотрено (кроме предусмотренных трудовым законодательством РФ и локальными актами Работодателя).
	
	\section{Ответственность сторон}
		В этом разделе сказано про материальную ответственность обеих сторон по отношению к другой.
	
	\section{Прекращение договора}
		Здесь определяется:
		\begin{enumerate}
			\item когда договор может быть прекращён (действующее законодательство Российской Федерации);
			
			\item какой день считать днём прекращения договора (последний день работы Работника, за исключением случаев, когда Работник фактически не работал).
		\end{enumerate}
	\section{Заключительные положения}
		В этом разделе указано, что споры между сторонами, возникающие при исполнении трудового договора, рассматриваются в порядке, установленном действующим законодательством Российской Федерации; что договор вступает в силу с момента его подписания.
	\section{Реквизиты сторон}
		В этом разделе описаны основные сведения, необходимые для заключения договора, такие как:
		\begin{enumerate}
			\item ФИО работника;
			
			\item адрес работника;
			
			\item паспортные данные работника;
			
			\item телефон работника;
			
			\item адрес электронной почты работника;
			
			\item счёт работника;
			
			\item сведения о работодателе (телефон, адрес электронной почты, ИНН/КПП, ОГРН, счёт, $\dots$).
		\end{enumerate}
\end{document}
